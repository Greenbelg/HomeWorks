\documentclass{article}
\usepackage[T1]{fontenc}
\usepackage[utf8]{inputenc}
\usepackage{amsmath,amssymb,amsfonts}
\usepackage[english,main=russian]{babel}
\usepackage{graphicx}
\graphicspath{{.}}

\begin{document}

\noindent \textbf{2. Исследовать на экстремум функцию:\\ $f(x,y)=x^2+xy+y^2-4\ln x - 10 \ln y$}\vspace{2mm}

\noindent$f'_x = 2x + y - \frac{4}{x} = 0$\vspace{1mm}

\noindent$f'_y = 2y + x - \frac{10}{y} = 0$\vspace{1mm}

\noindent Выражая $x$ из второго уравнения и решая уравнение с $y$, получаем такие возможные точки экстремума: \vspace{1mm}

\noindent$(x_0,y_0)\in\{(-1,-2), (1,2), (-\frac{4\sqrt{3}}{3},\frac{5\sqrt{3}}{3}), (\frac{4\sqrt{3}}{3},-\frac{5\sqrt{3}}{3})\}$. Но из-за ООФ логарифма точки, в которых хотя бы одна из координат $\leq 0$, не подходят, поэтому возможная точка всего одна - $(x_0,y_0) = (1,2)$.\vspace{1mm}

\noindent$f''_{xx} = 2 + \frac{4}{x^2}, f''_{xx}(1,2) = 6$\vspace{1mm}

\noindent$f''_{yy} = 2 + \frac{10}{y^2}, f''_{yy} (1,2) = 4.5$\vspace{1mm}

\noindent$f''_{xy} = f''_{yx} = 1, f''_{xy}(1,2) = 1$\vspace{1mm}

\noindentВоспользуемся достаточным условием экстремума в точке:\vspace{1mm}

\noindent$Q(a_1,a_2) = f''_{xx}(1,2)a_1a_1 + 2f''_{xy}(1,2)a_1a_2 + f''_{yy}(1,2)a_2a_2 =$\vspace{1mm}

\noindent$= 6a_1a_1 + 2a_1a_2 + 4.5a_2a_2$\vspace{1mm}

\noindent$\begin{vmatrix}
6 & 1 \\ 
1 & 4.5
\end{vmatrix} = 26$\vspace{1mm}

\noindent По критерию Сильвестра такая квадратичная форма положительно определена $\implies (1,2)$ - точка локального минимума.\vspace{3mm}

\noindent \textbf{3.3. Найти экстремумы функции $u = u(x,y)$, заданной неявно}\vspace{2mm}

\noindent $x^2+y^2+u^2-xu-yu+2x+2y+2u-2=0$ - продифференцируем обе части\vspace{1mm}

\noindent $2xdx + 2ydy + 2udu - xdu - udx - ydu - udy + 2dx + 2dy + 2du = 0$\vspace{1mm}

\noindent $(2x-u+2)dx + (2y-u+2)dy + (2u - x - y + 2)du = 0$\vspace{1mm}

\noindent$du = \frac{u-2-2x}{2u-x-y+2}dx + \frac{u-2-2y}{2u-x-y+2}dy$\vspace{1mm}

\noindent Из необходимого условия точки экстремума следует, что\vspace{1mm}

\noindent $\frac{u-2-2x}{2u-x-y+2}=0$ и $\frac{u-2-2y}{2u-x-y+2} = 0$\vspace{1mm}

\noindent$x = \frac{u}{2} - 1$\vspace{1mm}

\noindent$y = \frac{u}{2} - 1$\vspace{1mm}

\noindent Подстановкой в исходное равенство $x,y$ находим точки, в которых возможен экстремум: \vspace{1mm}

\noindent $u(x_0,y_0)=-4-2\sqrt{6} \implies x_0 = y_0 = -3-\sqrt{6}$\vspace{1mm}

\noindent$u(x_1,y_1)=-4+2\sqrt{6} \implies x_1=y_1 = -3 + \sqrt{6}$\vspace{1mm}

\noindent Попробуем воспользоваться достаточным условием:\vspace{1mm}

\noindent $d((2x-u+2)dx + (2y-u+2)dy + (2u - x - y + 2)du) = 0$\vspace{1mm}

\noindent $[2dx^2 - dudx] + [2dy^2 - dudy] + [2du^2 - dxdu - dydu] + (2u-x-y+2)d^2u= 0$\vspace{1mm}

\noindent $[\text{Т.к. du = 0}]d^2u = \frac{2}{x+y-2-2u}dx^2 + \frac{2}{x+y-2-2u}dy^2$\vspace{10mm}

\noindent Проверим найденные точки:
\begin{enumerate}
\item $u(x_0,y_0) = -4-2\sqrt{6}: d^2u = \frac{2}{[-3-\sqrt{6}]+[-3-\sqrt{6}]-2-2[-4-2\sqrt{6}]}dx^2 +\frac{2}{[-3-\sqrt{6}]+[-3-\sqrt{6}]-2-2[-4-2\sqrt{6}]}dy^2=$\\
$=\frac{1}{\sqrt{6}}dx^2+\frac{1}{\sqrt{6}}dy^2$\\
$Q_{x_0y_0}=\frac{1}{\sqrt{6}}a_1^2+\frac{1}{\sqrt{6}}a_2^2$ - положительно определена, значит, это точка локального минимума;
\item $u(x_1,y_1) = -4 + 2\sqrt{6}: d^2u = -\frac{1}{\sqrt{6}}dx^2 -\frac{1}{\sqrt{6}}dy^2$\\
$Q_{x_1y_1}=-\frac{1}{\sqrt{6}}a_1^2-\frac{1}{\sqrt{6}}a_2^2$ - отрицательно определена, значит, это точка локального максимума.
\end{enumerate}
\end{document}