\documentclass{article}
\usepackage[T1]{fontenc}
\usepackage[utf8]{inputenc}
\usepackage{amsmath,amssymb,amsfonts}
\usepackage[english,main=russian]{babel}
\usepackage{graphicx}
\graphicspath{{.}}

\begin{document}
\noindent \textbf{7. В урне 4 белых и 7 черных шаров. Из урны наудачу один за другим извлекают два шара, не возвращая их обратно. Найти вероятность того, что:}

\begin{itemize}
\item оба шара будут белыми: всего из урны можно выбрать $C^2_4$ пар белых шаров, а общее количество пар всех шаров будет $C^2_{11}$, т.к. порядок выбора шаров в паре не важен. Поэтому вероятность равна $\frac{C^2_4}{C_{11}^2} = \frac{6}{55}$;

\item оба шара будут чёрными: аналогичные рассуждения. Вероятность будет равна $\frac{C^2_7}{C_{11}^2} = \frac{21}{55}$;

\item сначала будет извлечён белый шар, а затем – чёрный: достоверное событие будет таким: из урны извлечены два белых шара или два черных шара или черный и белый шары(без учета порядка). Поэтому вероятность того, что из урны извлечены черный и белый шары будет $1 - \frac{6}{55} - \frac{21}{55} = \frac{28}{55}$. И вероятность события $"$сначала будет извлечён белый шар, а затем – чёрный$"$ - половина от найденной вероятности, т.е. равна $\frac{14}{55}$.
\end{itemize}

\noindent \textbf{8. Вероятность того, что стрелок поразит мишень при выстреле из винтовки с оптическим прицелом, равна 0,95; для винтовки без оптического прицела эта вероятность равна 0,7. Найти вероятность того, что мишень будет поражена, если стрелок производит один выстрел из наудачу взятой винтовки. Вероятность взять винтовку с оптическим прицелом — 0,6}

Пусть A - вероятность поражения мишени при выстреле из винтовки с оптическим прицелом, B - вероятность поражения мишени при выстреле из винтовки без оптического прицела и С - вероятность взять винтовку с оптическим прицелом. Тогда из условия знаем, что $P(A|C) = 0.95, P(B|\overline{C}) = 0.7, P(C) = 0.6$. Поэтому из формул полной вероятности и условной вероятности получаем: вероятность того, что мишень будет поражена = $0.95 * 0.6 + 0.7 * (1 - 0.6) = 0.85$.

\noindent \textbf{9. Расследуются причины авиационной катастрофы, о которых можно сделать четыре гипотезы: $B_1, B_2, B_3, B_4$. Согласно статистике вероятности гипотез составляют: 
\begin{itemize}
\item $P(B_1) = 0,2$
\item $P(B_2) = 0,4$
\item $P(B_3) = 0,3$
\item$P(B_4) = 0,1$
\end{itemize}
    Осмотр места катастрофы выявляет, что в её ходе произошло событие $A$ — воспламенение горючего. Условные вероятности события, согласно той же статистике равны:
\begin{itemize}
\item$P(A|B_1) = 0,9$
\item$P(A|B_2) = 0$
\item$P(A|B_3) = 0,2$
\item$P(A|B_4) = 0,3$
\end{itemize}
    Найти апостериорные вероятности гипотез.}

Необходимо найти $P(B_1|A), P(B_2|A), P(B_3|A), P(B_4|A)$. Для начала найдем вероятность события $A$ из формулы условной вероятности:

$P(AB_i) = P(A|B_i)*P(B_i)$, $\forall i \in \{1,2,3,4\}$, 

$\sum^4_{i=1}P(AB_i) = P(A(\cup_{i=1}^4B_i)) = P(A)$,

значит $P(A) = \sum^4_{i=1}P(A|B_i)*P(B_i) = 0.18 + 0 + 0.06 + 0.03 = 0.27$.

Чтобы найти апостериорные вероятности гипотез, воспользуемся формулой Байеса: $P(B|A)=\frac {P(A|B)P(B)}{P(A)}$. Получается,
\begin{itemize}
\item $P(B_1|A) = \frac {P(A|B_1)P(B_1)}{P(A)} = \frac{0.18}{0.27} = \frac{2}{3},$
\item $P(B_2|A) = \frac {P(A|B_2)P(B_2)}{P(A)} = 0,$
\item $P(B_3|A) = \frac {P(A|B_3)P(B_3)}{P(A)} = \frac{0.06}{0.27} = \frac{2}{9},$
\item $P(B_4|A) = \frac {P(A|B_4)P(B_4)}{P(A)} = \frac{0.03}{0.27} = \frac{1}{9}$.
\end{itemize}
\end{document}