\documentclass{article}
\usepackage[T1]{fontenc}
\usepackage[utf8]{inputenc}
\usepackage{amsmath,amssymb,amsfonts}
\usepackage[english,main=russian]{babel}
\usepackage{graphicx}
\graphicspath{{.}}

\begin{document}
\noindent \textbf{1. Пусть $A_1, A_2, \dots, A_n, \dots$ некоторые подмножества $\Omega$, постройте минимальную $\sigma$-алгебру, включающую $A_1, A_2, …, A_n, …$}

Для построения такой алгебры (назовем ее $F$) выполним минимальные требования:

\begin{enumerate}
\item $\emptyset \in F \implies F = \{\emptyset, A_1, \dots, A_n, \dots\}$. 

\item $A \in F \implies \overline{A} \in F$. $F = \{\emptyset, \Omega, A_1, \overline{A_1}, \dots, A_n, \overline{A_n}, \dots\}$. 

\item $A_1, A_2, \dots \in F \implies \cup_{i=1}^{\infty}A_i \in F$. Т.е. в $F$ войдут всевозможные 2, 3, $\dots$, n, $\dots$ элементные объединения различных подмножеств из $A_1, \dots, A_n, \dots$.
\end{enumerate}

\noindent \textbf{2. Доказать, что алгебра, порожденная системой $A_1, \dots, A_n$, где $A_i \subset \Omega$,  $i=1,\dots,n$ состоит в общем случае из $2^{2^n}$ элементов. Найти пример системы множеств, когда это не так.}

Как показано в предыдущей задаче, такая алгебра будет в себя включать $\emptyset, \Omega, A_1, A_2, \dots, A_n$ и всевозможные 2, 3, $\dots$, n элементные объединения различных подмножеств из $A_1, \dots, A_n$. 
\end{document}