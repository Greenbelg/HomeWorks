\documentclass{article}
\usepackage[T1]{fontenc}
\usepackage[utf8]{inputenc}
\usepackage{amsmath,amssymb,amsfonts}
\usepackage[english,main=russian]{babel}
\usepackage{graphicx}
\graphicspath{{.}}

\begin{document}
\noindent \textbf{1. Пусть $A_1, A_2, \dots, A_n, \dots$ некоторые подмножества $\Omega$, постройте минимальную $\sigma$-алгебру, включающую $A_1, A_2, …, A_n, …$}

Для построения такой алгебры (назовем ее $F$) выполним минимальные требования:

\begin{enumerate}
\item $\emptyset \in F \implies F = \{\emptyset, A_1, \dots, A_n, \dots\}$. 

\item $A \in F \implies \overline{A} \in F$. $F = \{\emptyset, \Omega, A_1, \overline{A_1}, \dots, A_n, \overline{A_n}, \dots\}$. 

\item $A_1, A_2, \dots \in F \implies \cup_{i=1}^{\infty}A_i \in F$. Т.е. в $F$ войдут всевозможные 2, 3, $\dots$, n, $\dots$ элементные объединения из $A_1, \overline{A_1}, \dots, A_n, \overline{A_n}, \dots$.
\end{enumerate}

\noindent \textbf{2. Доказать, что алгебра, порожденная системой $A_1, \dots, A_n$, где $A_i \subset \Omega$,  $i=1,\dots,n$ состоит в общем случае из $2^{2^n}$ элементов. Найти пример системы множеств, когда это не так.}

Как показано в предыдущей задаче, такая алгебра будет в себя включать $\emptyset, \Omega, A_1, A_2, \dots, A_n$ и всевозможные 2, 3, $\dots$, n элементные объединения различных подмножеств из $A_1, \dots, A_n$. 


\noindent \textbf{3. Сейчас либо солнечно, либо дождь, либо пасмурно без дождя. Соответственно множество $\Omega$ состоит из трёх исходов, $\Omega = \{солнечно, дождь, пасмурно\}$. Джеймс Бонд пойман и привязан к стулу с завязанными глазами, но он может на слух отличать, идет ли дождь.}

\begin{enumerate}
\item \textbf{Как выглядит $\sigma$-алгебра событий, которые различает агент 007?}

Агент может знать: идет дождь, не идет дождь - значит, $\sigma$-алгебра будет иметь вид $\{\text{дождь}, \{\text{пасмурно, солнечно}\}, \emptyset, \Omega\}$.

\item \textbf{Как выглядит минимальная алгебра, содержащая $A=\{\emptyset\}$?}

Пусть алгебра содержит $A$. Тогда она должна содержать дополнение к $A - \Omega$ и их объединение - $\Omega$. 
Значит, минимальная алгебра, содержащая $A$, будет $\{\emptyset, \Omega\}$.

\item \textbf{Сколько различных $\sigma$-алгебр можно придумать для данного $\Omega$?}

\begin{enumerate}
\item Первое такое множество построено в первом пункте;
\item Второе - во втором;
\item Третью $\sigma$-алгебру можно построить, если предположить, что агент умеет различать только солнечно на улице или нет (например, потому что он стал вампиром). Тогда $\sigma$-алгебра будет иметь вид $\{\text{солнечно}, \{\text{пасмурно, дождь}\}, \emptyset, \Omega\}$;
\item Четвертая $\sigma$-алгебра строится аналогично первой и третьей - агент знает только то, что на улице пасмурно. Она будет такой:

\{пасмурно, \{солнечно, дождь\}, $\emptyset, \Omega\}$;
\item Предположим, что агент знает уже два состояние погоды - (Б.О.О.) солнечно или пасмурно. Тогда в $\sigma$-алгебру попадет их пересечение - \{пасмурно, солнечно\} - и дополнение к пересечению - \{дождь\}. Значит, зная какие-то 2 состояния погоды, агент автоматически будет знать и 3-е. Поэтому это последняя различная $\sigma$-алгебра, которую можно построить на $\Omega:$ 

\{солнечно, \{пасмурно, дождь\}, дождь, \{пасмурно, солнечно\}, пасмурно, \{солнечно, дождь\}, $\emptyset, \Omega\}$
\end{enumerate}
Значит, таких монжеств всего 5.
\end{enumerate}

\noindent \textbf{4. Монеточка подкидывается бесконечное число раз: $X_n$ равно 1, если при $n$-ом подбрасывании выпал орел, и 0, если решка. И ничего другого выпасть не может. Определим несколько $\sigma$-алгебр: $F_n=\sigma(X_1,X_2,…,X_n), H_n:=\sigma(X_n,X_{n+1},X_{n+2},…)$}
\begin{enumerate}
\item \textbf{Приведите по два нетривиальных (т.е. $\Omega$ и $\emptyset$ не называть) примера такого события $A$, что:}
\begin{itemize}
\item $A \in F_{2021}$: В $F_{2021}$ будет содержаться событие "при первом подбросе монетки выпал орёл" т.е. $X_1 = 1$, т.к. если это не так, то дополнение к "при первом подбросе монетки выпала решка" всё равно будет лежать в сигма-алгебре - это пример первого события;

Также в $F_{2021}$ будет событие "в промежутке между 42 и 101 всегда выпадал орел" т.е. $\{X_{42}=1, X_{43} = 1, \dots, X_{101} = 1\}$, т.к. объединение этих событий, либо их дополнений, должно лежать в сигма-алгебре - это 2 пример; 
\item $A \notin F_{2021}$: В данном случае нам известны исходы на 2021 подбрасываниях, но ничего не известно насчет $X_{2023} = A$ - поэтому это пример такого события;

Также ничего не известно о множестве $\{X_{2024}, X_{2024}, \dots, X_{2042}\} = A$ - это второй такой пример; 
\item $A$ лежит в каждой $H_n$: событие "хотя бы раз выпала решка" лежит в каждой $H_n$, т.к. даже если всегда выпадал орел, дополнение к каждому из этих событий - выпала решка - должно быть в $\sigma$-алгебре;

Также событие "решка выпала бесконечное число раз"  лежит в каждой $H_n$, т.к. монетка подбрасывается бесконечное количество раз и если откинуть конечное число $n$ брасаний на бесконечность это никак не повлияет.
\end{itemize}
\item \textbf{В какие из упомянутых $\sigma$-алгебр входят события:}
\begin{itemize}
\item $X_{45}>0$: $F_n$, $\forall n \in N \backslash \{1,2,\dots,44\}$; $H_n$, $\forall n \in \{1,2,\dots,45\}$;
\item $X_{45}>X_{2021}$: $F_n$, $\forall n \in N \backslash \{1,2,\dots,2020\}$; $H_n$, $\forall n \in \{1,2,\dots,45\}$;
\item $X_{45}>X_{2021}>X_{15}$: $min(X_{15})=0 \land X_{2023} \in \{0,1\} \implies X_{2023} = 1, X_{45} > 1 > 0$, но $X_{45} \in \{0,1\} \implies$ такое событие не могло случиться, значит оно невозможное, т.е. это $\emptyset$. $\emptyset$ входит в любую $\sigma$-алгебру.
\end{itemize}
\item \textbf{Упростите выражения:}
\begin{itemize}
\item$F_{11}\cap F_{25}$ = $\sigma(X_1,X_2,…,X_{11}) \cap \sigma(X_1,X_2,…,X_{25}) = \sigma(X_1,X_2,…,X_{11}) = F_{11}$ - потому что $F_{11}$ полностью содержится в $F_{25}$ из построения $\sigma$-алгебры;
\item $F_{11} \cup F_{25}$ = $F_{25}$ - потому что $F_{11}$ полностью содержится в $F_{25}$ из построения $\sigma$-алгебры;
\item $H_{11} \cap H_{25}$ = $\sigma(X_{11},X_{12},X_{13},…, X_{25}, X_{26}, \dots) \cap \sigma(X_{25},X_{26},X_{27},…) = H_{25}$ - из построяние следует, что $H_{25}$ лежит в $H_{11}$;
\item $H_{11} \cup H_{25} = H_{11}$.
\end{itemize}
\end{enumerate}

\end{document}