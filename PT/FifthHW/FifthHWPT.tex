\documentclass{article}
\usepackage[T1]{fontenc}
\usepackage[utf8]{inputenc}
\usepackage{amsmath,amssymb,amsfonts}
\usepackage[english,main=russian]{babel}
\usepackage{graphicx}
\graphicspath{{.}}

\begin{document}

\noindent \textbf{1. Стрелок сделал 30 выстрелов с вероятностью попадания при отдельном выстреле 0,3. Найти вероятность того, что при этом будет 8 попаданий.}

Обозначим за $p=0,3$ - вероятность попадания, тогда $q=1-p=0,7$ - вероятность промаха. Воспользуемся формулой $"$Биномиальное распределение$"$: $P_{30}(8) = C_{30}^8p^8q^{22}= 5852925 * 0.00006561 * 0.0003909821 \approx 0.150141$ - ответ на заданную задачу.

\noindent \textbf{2. Вероятность попадания в цель при одном выстреле равна 0,001. Для поражения цели необходимо не менее двух попаданий. Произведено 5000 выстрелов. Найти вероятность поражения цели.}

Найдем вероятность того, что из 5000 выстрелов в мишень было ровно одно попадание или попаданий не было вовсе: обозначим за $p=0.001$ - вероятность попадания в мишень и $q = 1 - p = 0.999$ - вероятность промаха. Тогда вероятность $"$попасть в мишень лишь раз за 5000 выстрелов$"$ вычисляется по формуле $"$Биномиальное распределение$"$: $P_{5000}(1) = C_{5000}^1 p q^{4999}=5000 * 0.001 * 0.999^{4999}$; вероятность $"$не попасть в мишень вовсе$"$ вычисляется по той же формуле: $P_{5000}(0) = C_{5000}^0 q^{5000} = 0.999^{5000}$. И вероятность $"$поражение мишени$"$ будет равна $1 - 5 * 0.999^{4999} - 0.999^{5000} = 1 - 0.999^{4999}*5.999 \approx 0.95964$.

Ответ: 0.95964.

\noindent \textbf{4. Некоторая машина состоит из 10 тысяч деталей. Каждая деталь независимо от других деталей может оказаться неисправной с вероятностью $p_i$, причём
\begin{itemize}
\item 1000 из всех деталей имеют вероятность неисправности $p_1$=0,0003;
\item 2000 из всех деталей $p_2$=0,0002;
\item 7000 оставшихся деталей $p_3$=0,0001.
\end{itemize}
    Машина не работает, если в ней неисправны хотя бы две детали. Найти вероятность того, что машина не будет работать.}

Найдем вероятность того, что машина работает: либо нет сломаных деталей, либо сломана только одна. Вероятность того, что в машине нет сломанных деталей, равна $(1-p_1)^{1000}(1-p_2)^{2000}(1-p_3)^{7000}$. Вероятность того, что в машине только одна деталь неисправна, $"$собирается$"$ из следующих возможных событий: одна из 1000 деталей сломана, остальные - рабочие, одна из 2000 деталей сломана, другие - рабочие, одна из 7000 деталей сломана, другие - рабочие. 

Рассмотрим последовательно каждое событие:
\begin{enumerate}
\item одна из 1000 деталей, которые имеют вероятность неисправности $p_1$, - сломана. Такая вероятность равна: $(C_{1000}^1p_1(1-p_1)^{999})(1-p_2)^{2000}(1-p_3)^{7000}$;
\item одна из 2000 деталей, которые имеют вероятность неисправности $p_2$, - сломана. Такая вероятность равна: $(C_{2000}^1p_2(1-p_2)^{1999})(1-p_1)^{1000}(1-p_3)^{7000}$;
\item одна из 7000 деталей, которые имеют вероятность неисправности $p_3$, - сломана. Такая вероятность равна: $(C_{7000}^1p_3(1-p_3)^{6999})(1-p_1)^{1000}(1-p_2)^{2000}$;
\end{enumerate}

Тогда вероятность того, что машина работает, равна: $(1-p_1)^{1000}(1-p_2)^{2000}(1-p_3)^{7000} + C_{1000}^1p_1(1-p_1)^{999}(1-p_2)^{2000}(1-p_3)^{7000} + C_{2000}^1p_2(1-p_2)^{1999}(1-p_1)^{1000}(1-p_3)^{7000} + C_{7000}^1p_3(1-p_3)^{6999}(1-p_1)^{1000}(1-p_2)^{2000} = ((1-p_1)^{999}(1-p_2)^{1999}(1-p_3)^{6999})((1-p_1)(1-p_2)(1-p_3) + 1000 * p_1(1-p_2)(1-p_3)+2000*p_2(1-p_1)(1-p_3)+7000*p_3(1-p_1)(1-p_2)) \approx 0.267342$

Тогда вероятность того, что машина не работает, равна $1 - 0.267342 =0.732658 $.

Ответ: 0.732658.

\noindent \textbf{5. У театральной кассы стоят в очереди $2n$ человек. Среди них $n $ человек имеют лишь банкноты по 1000 рублей, а остальные — только банкноты по 500 рублей. Билет стоит 500 рублей. Каждый покупатель приобретает по одному билету. В начальный момент в кассе нет денег. Чему равна вероятность того, что никто не будет ждать сдачу?}

Решим сначала такую задачу: сколько существует различных порядков в очереди, таких, что кассир всегда может сдать сдачу. В курсе $"$Дискретная математика$"$ было доказано, что число таких порядков - в точности число Каталана: $C_n = \frac{1}{n+1}C_{2n}^n$. 

Тогда ответ на заданную задачу - отношение $"$хороших$"$ порядков в очереди, показанныз выше, к общему числу очередей - их число $(2n)!$. Тогда вероятность будет равна: $\frac{C_{2n}^n}{(n+1)(2n)!} = \frac{(2n)!}{n!n!(n+1)(2n)!}= \frac{1}{(n+1)!n!}$
\end{document}