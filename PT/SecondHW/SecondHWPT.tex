\documentclass{article}
\usepackage[T1]{fontenc}
\usepackage[utf8]{inputenc}
\usepackage{amsmath,amssymb,amsfonts}
\usepackage[english,main=russian]{babel}
\usepackage{graphicx}
\graphicspath{{.}}

\begin{document}

\noindent \textbf{1. Пусть есть множество множеств $\{\{1,2,3,4\},\{3,4,5,6\}\}$. Дополните эту систему до:}

\begin{enumerate}

\item(Минимального полукольца) В полукольце должно быть пустое множество - $\emptyset \implies$ множество становится $\{\{1,2,3,4\},\{3,4,5,6\}, \emptyset\}$

В полукольце есть пересечение любых двух элементов $\implies$ множество становится $\{\{1,2,3,4\},\{3,4,5,6\}, \emptyset, \{3,4\}\}$

В полукольце для каждого элемента есть разбиение на элементы из полукольца $\implies$ множество становится 

$\{\{1,2,3,4\},\{3,4,5,6\}, \emptyset, \{3,4\}, \{1,2\}, \{5,6\}\}$
 - это минимальное полукольцо.

\item(Минимального кольца) Любое кольцо - полукольцо. 

Значит $\{\{1,2,3,4\},\{3,4,5,6\}, \emptyset, \{3,4\}, \{1,2\}, \{5,6\}\}$ содержится и в кольце.

Для любых двух элементов из кольца в кольце содержится их симметрическая разность $\implies$ множество становится 

$\{\{1,2,3,4\},\{3,4,5,6\}, \emptyset, \{3,4\}, \{1,2\}, \{5,6\}, \{1,2,5,6\}, \{1,2,3,4,5,6\}\}$


\item(Минимальной алгебры) Полученное нами кольцо

$\{\{1,2,3,4\},\{3,4,5,6\}, \emptyset, \{3,4\}, \{1,2\}, \{5,6\}, \{1,2,5,6\}, \{1,2,3,4,5,6\}\}$ 

является алгеброй.

В данном случае единицей является элемент $\{1,2,3,4,5,6\}$ - он лежит во множестве и пересечение любого элемента с ним дает сам элемент.
\end{enumerate}

\noindent \textbf{2. Доказать, что:}

\begin{enumerate}
\item Пересечение произвольной непустой системы колец является кольцом (возможно, состоящим лишь из пустого множества):

Проверим, что это кольцо, опираясь на определение кольца:

\begin{enumerate}
\item Если во множестве есть элементы $A, B$, то в нем есть $A \cap B$:

От противного: предположим, что в пересечение попали элементы $A, B$, но не $A \cap B$. Т.к. $A, B$ лежат в пересечении, они принадлежат каждому из колец. Но если $A, B$ лежат в кольце, то и $A \cap B$ лежит в кольце. Значит, в каждом кольце есть $A \cap B \implies$ в пересечение колец есть$A \cap B$. Получаем противоречие.

\item  Если во множестве есть элементы $A, B$, то в нем есть $A \triangle B$:

Также как и с предыдущим подпунктом - преположением от противного приходим к выводу, что если $A, B$ лежат в пересечение, то и  $A \triangle B$ лежит в пересчение колец.

\item Отдельным пунктом выделим, что система не пуста:

Т.к. кольцо является и полукольцом, в нем есть $\emptyset$. Значит, и в пересечении колец будет лежать хотя бы $\emptyset$, ведь оно есть в каждом кольце.
\end{enumerate}

\item Пересечение произвольной непустой системы $\sigma$-колец является $\sigma$-кольцом:


\item Пересечение непустой (конечной) системы алгебр с одной и той же единицей является алгеброй:

Алгебра - кольцо с единицей. Уже доказано выше, что пересечение колец - кольцо. Т.к. единица(назовем ее $E$) в каждой алгебре одна и та же, она войдет и в пересечение. И в итоговом кольце $E$ также будет едининцей:

От противного: если $\exists A \text{ из полученного кольца, такой что } A \cap E \neq A$, то этот элемент принадлежит и всем алгебрам. Но в них $A \cap E = A$ - что войдет и в пересечние. Получаем противоречие.

\end{enumerate}

\noindent \textbf{2. Доказать, что:}

\end{document}