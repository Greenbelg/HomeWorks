\documentclass{article}
\usepackage[T1]{fontenc}
\usepackage[utf8]{inputenc}
\usepackage{amsmath,amssymb,amsfonts}
\usepackage[english,main=russian]{babel}
\usepackage{graphicx}
\graphicspath{{.}}

\begin{document}

\noindent \textbf{1. Пусть есть множество множеств $\{\{1,2,3,4\},\{3,4,5,6\}\}$. Дополните эту систему до:}

\begin{enumerate}

\item(Минимального полукольца) В полукольце должно быть пустое множество - $\emptyset \implies$ множество становится $\{\{1,2,3,4\},\{3,4,5,6\}, \emptyset\}$

В полукольце есть пересечение любых двух элементов $\implies$ множество становится $\{\{1,2,3,4\},\{3,4,5,6\}, \emptyset, \{3,4\}\}$

В полукольце для каждого элемента есть разбиение на элементы из полукольца $\implies$ множество становится 

$\{\{1,2,3,4\},\{3,4,5,6\}, \emptyset, \{3,4\}, \{1,2\}, \{5,6\}, \{1\}, \{2\}, \{3\}, \{4\}, \{5\}, \{6\}\}$ - это минимальное полукольцо.

\item(Минимального кольца) Любое кольцо - полукольцо. 

Значит $\{\{1,2,3,4\},\{3,4,5,6\}, \emptyset, \{3,4\}, \{1,2\}, \{5,6\}, \{1\}, \{2\}, \{3\}, \{4\}, \{5\}, \{6\}\}$ содержится и в кольце.

Для любых двух элементов из кольца в кольце содержится их симметрическая разность $\implies$ множество становится 

$\{\{1,2,3,4\},\{3,4,5,6\}, \emptyset, \{3,4\}, \{1,2,5,6\}\}$
\item(Минимальной алгебры)

\end{enumerate}

\end{document}